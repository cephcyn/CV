\begin{rubric}{Research Experience}

\entry*[2019 -- $\cdots\cdot$] % 2019aug - 2020...
	\textbf{Graduate Researcher}, \href{http://hai.cs.washington.edu}{Lab for Human-AI Interaction}
	\par Mentored by \pBansalG{} and advised by \pWeldD{}.
	% \belinda{Describe your role in the research: Worked on /blah blah blah investigating AI explanations/}
	\par Developed, implemented, and evaluated a novel adaptive explanation style for human-AI teams on a sentiment analysis task. Worked on analyzing participants' feedback on how AI suggestions and explanations factored into their decision-making.
	\par Resulted in joint 2nd-author publication and submission to CSCW, see \cite{bansal2020does}. Also featured in a WHI 2020 spotlight.
\entry*[2018 -- 2019] % 2018sep – 2019aug
    \textbf{Undergraduate Researcher} with \pRuzzoL{}
    \par Developed a set of tools (\textit{blockmerge} and \textit{crosscompare}) and a pipeline centered on CMfinder to search for potentially structured fRNA sequences across alignment block boundaries and cluster found covariance models.
    \par Wrote up methods and findings in Bachelor's thesis, see \cite{zhou2019thesis}.
% \belinda{indicates that your work actually produced something and that you had the follow-through to make that happen, instead of dead-ending and getting dropped (which is often the case in undergrad research)}
\entry*[2018 -- 2019] % 2018mar - 2019jun
	\textbf{Undergraduate Researcher} with \pPahnkeE{} (Foster School of Business, UW)
	\par Collected, organized, and cleaned data from a diverse range of websites (social media, blogs, business homepages) to form an original data set.
\end{rubric}